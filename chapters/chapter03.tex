\section{Stability of fixed points}
\label{chap03}

\subsection{Basic definitions}

\textbf{Consider}: $\dot{x}=f(x,t), \quad x \in R^n, \quad f \in C^1$\\
\textbf{Assume}: $x=0$ is a fixed point, i.e., $f(0,t)=0, \quad \forall t \in R$ 
\textbf{Question}: How does the dynamical system behave near its equilibrium state?

\subsubsection{Stability, due Lyapunov}
$x=0$ is stable if $\forall t_0, \forall \epsilon > 0 \text{ (small enough)}, \exists \delta=\delta(t_0, \epsilon)$, such that for $\forall x_0 \in R^n \text{ with } |x_0| 	\leq \delta$, we have:
$$|x(t;t_0,x_0)|\leq\epsilon, \quad \forall t \geq t_0$$

\subsubsection{Asymptotic stability}
$x=0$ is asymptotically stable if:
\begin{itemize}
    \item it is stable
    \item $\forall t_0, \quad \exists \delta_0(t_0)$ such that $\forall x_0: \quad |x_0|\leq \delta_0$: \\
    $$\lim_{t\to\infty} x(t;t_0, x_0)=0$$
\end{itemize}

\subsubsection{Domain of attraction (fixed point assumed to be 0)}
Set of all $x_0$'s for which: $x(t;t_0,x_0)\to0, \text{ as } t\to\infty$

\textbf{Example 2}: Damped oscillator\\
Check out page 32 of PDF with notes. To understand how he derives the rate of energy change, first check out how he defines $x_1$ and $x_2$ and then take the derivative of the energy E with respect to time:
\begin{align}
    \frac{dE}{dt}=\frac{1}{2}\cdot2x_2\cdot\dot{x_2}+\sin x_1\cdot x_2=x_2(\dot{x_2}+\sin x_1) = x_2(-cx_2-\sin x_1 + \sin x_1)=-cx_2^2  
\end{align}

\subsubsection{Invariant set}
S $\subset$ P is an invariant set for the flow map $F^t: p \to p$ if $F^t(S)=S, \forall t\in R$ (meaning, if I take an initial condition inside the set, I stay in the set).

\subsubsection{Instability}
$x=0$ is unstable if it is not stable.

\section{Stability based on linearization (for autonomous systems)}
We are in a multivariate setting! $x$ is a vector, and so is f and fixed point p.
\begin{align}
    \dot{x}=f(x), \quad x \in R^n, \quad f \in C^1
\end{align}
\textbf{Question:} How can we systematically derive/determine the stability of the fixed point of the preceding system?\\

We assume that $f(p)=0 \Rightarrow y=x-p$
ODE (1) in transformed coordinates: $$\dot{y}=f(p+y)=f(p)+Df(p)y+O(|y|^2)$$
$$\dot{y}=Df(p)y+O(|y|^2)$$
Define the linearization of (1) at the fixed point p:
$$\dot{y}=Ay, \quad y\in R^n, \quad A:=Df(p) \in R^{n\times n}$$
\[
Df(p)=
  \begin{bmatrix}
    \frac{\partial f_1}{\partial x_1} & \frac{\partial f_1}{\partial x_2} & ... & \frac{\partial f_1}{\partial x_n} \\
    .\\
    .\\
    .\\
    \frac{\partial f_n}{\partial x_1} & \frac{\partial f_n}{\partial x_2} &...& \frac{\partial f_n}{\partial x_n}
  \end{bmatrix}
\]

\textbf{To study:}
\begin{itemize}
    \item stability of y=0 in $\dot{y}=Ay$
    \item relevance of this for the nonlinear system defined in the beginning of this section
\end{itemize}

\subsection{Review of linear dynamical systems}
$$\dot{y}=A(t)y, \quad y\in R^n, \quad A \in R^{n\times n}, \quad A\in C^0_t$$
\begin{itemize}
    \item Global existence and uniqueness guaranteed
    \item Superposition principle: linear combination of solutions is also a solution
    \item $\exists$ a set of n linearly independent solutions: $[\phi_1(t), \phi_2(t), ..., \phi_n(t)]$
    \item General soluton: $y(t)=\sum_{i=1}^n c_i\phi_i$ \\
    $\Rightarrow y(t)=[\phi_1(t), \phi_2(t), ..., \phi_n(t)]\cdot[c_1, ... c_n] = \Psi(t)c$ \\
    Where $\Psi$ is the \textbf{fundamental matrix solution}\\
    Hence: $\dot{= \Psi(t)}= A\Psi(t)$
    \item IVP: $$y(t_0)=y_0 \Rightarrow \Psi(t_0)c=y_0 \Rightarrow y(t)=\Psi(t)\cdot [\Psi(t_0)]^{-1}y_0$$
    Where $\Psi(t)\cdot [\Psi(t_0)]^{-1} = \phi(t)=F_{t_0}^t$
    $\phi$ is the normalized fundamental matrix, which is the flow map.
    $$\phi(t_0)=I$$
    \item Practical construction of solutions
        \begin{itemize}
            \item Autonomous case: $\dot{x}=Ax$ only!
            \item Explicit solution: aqui andámos a brincar com $\phi$s e com McLaurin series da função exponencial.
            \item Solution from eigenfunctions
        \end{itemize}
\end{itemize}

\subsection{Stability of fixed points in autonomous linear systems}
 \begin{theorem}
 \textbf{Stability of y=0 in the linear system $\dot{y}=Ay$}
    \begin{enumerate}
        \item Assume $Re(\lambda_j)<0, \quad \forall j$\\
        Then $y=0$ is asymptotically stable.
        \item Assume $Re(\lambda_j)\leq0, \quad \forall j$ and $\forall\lambda_k: Re\lambda_k=0$, we have $a_k=g_k$\\
        Then $y=0$ is stable.
        \item Assume that $\exists k: Re(\lambda_k)>0$\\
        Then $y=0$ is unstable.
    \end{enumerate}
 \end{theorem}
 
 \subsection{Stability of fixed points in non-linear systems}
 Consider two dynamical (autonomous) systems:
 \begin{enumerate}
     \item $\dot{x}=f(x), x\in R^n, f \in C^1 \Rightarrow F^t: x_0 \to x(t;x_0)$
    \item $\dot{x}=g(x), x\in R^n, g \in C^1 \Rightarrow G^t: x_0 \to x(t;x_0)$
 \end{enumerate}
 
 \begin{definition}
 The dynamical systems (1) and (2) are $C^k$ equivalent ($k\in\N$) on an open set $U\in\R^n$, if $\exists$ a $C^k$-diffeomorphism $h: U\to U$ that maps orbits of (1) into orbits of (2), preserving orientation but not becessarily the exact parameterization of the orbit by time.\\
 Specifically: $\forall x\in U, \forall t \in \R: \exists t_2 \in\R$ such that:
 $$h(F^{t_1}(x))=G^{t_2}(h(x))$$
 \end{definition}
 
 \begin{definition}
 For $k=0$, $C^k$-equivalence is also called topological equivalence, meaning a continuous invertible map that takes orbits of one system to orbits of the othre one. Such an $h: U\to U$ is called homeomorphism (and the two sets are called homeomorphic sets).
 \end{definition}
 
 \begin{quote}
     “Here topologically equivalent means that there is a homeomorphism (a continuous deformation with a continuous inverse) that maps one local phase portrait onto the other, such that trajectories map onto trajectories and the sense of time (the direction of the arrows) is preserved.
    \textbf{Intuitively, two phase portraits are topologically equivalent if one is a distorted version of the other. Bending and warping are allowed, but not ripping, so closed orbits must remain closed, trajectories connecting saddle points must not be broken, etc.}”
Excerpt From: Steven H. Strogatz. “Nonlinear Dynamics and Chaos: With Applications to Physics, Biology, Chemistry, and Engineering”. Apple Books. 
 \end{quote}
 
 \begin{definition}
 $x=x_0$ is a hyperbolic fixed point of (1) if the eigenvalues of its linearization (2) satisfy $Re(\lambda_i)\neq0, \quad \forall i=1,...,n$
 \end{definition}
 
 \begin{theorem}
 \textbf{Implicit function theorem}\\
 $\exists !$ nearby solution to $F(x,\epsilon)=0$ of the form  $x_{\epsilon}=x_0+O(\epsilon)$, provided that $D_xF(x_0, 0)$ is non-singular and $F(x,\epsilon)\in C^0$ ($x_{\epsilon}$ is as smooth in $\epsilon$ as $F(x,\epsilon)$) 
 \end{theorem}
 
 \textbf{Claim:} the linearized stability type of a hyperbolic fixed point is preserved under small perturbations to the linear system ($\dot{x}=f(x)$). Again this is more than just persisting hyperbolicity. Stability type is also preserved.\\
 In contrast, for non-hyperbolic fixed points, the smallest perturbation may change their stability type.
 
 \begin{theorem}
 \textbf{Hartman-Grobman}: if the fixed point $x_0$ of the nonlinear system (1) is hyperbolic, then (1) is topologically equivalent to its linearization (2) in a neighborhood of $x_0$.
 \end{theorem}
 \textbf{Consequence:} for hyperbolic fixed points, linearization predicts the correct stability type AND orbit geometry near $x_0$.\\
 
 \textbf{Consequence 2:} if you have eigenvalues with zero real part (they lie on the imaginary axis), then the fixed point is not hyperbolic and you cannot apply Hartman-Grobman.
 
 \subsubsection{A sufficient and necessary condition for $Re(\lambda_i) <0, \forall i$}
 \textbf{It's the Routh-Hurwitz stability criterion!} Check out the introduction of the corresponding Wikipedia article if in doubt.\\
 The procedure is to define a series of determinants.
 $$Re(\lambda_i)<0 \iff D_i>0, \quad 1\leq i\leq n$$
 
 \subsection{Lyapunov's direct (2nd) method for stability}

How to establish stability without reliance on linearization?
$$\dot{x}=f(x), \quad f\in C^1, \quad x \in \R^n, \quad \dot{x_0}=f(x_0)=0$$
Assume: $\exists V: U \to \R, \quad V \in C^1(U), \quad U \subset \R^n \text{open}, \quad x_0 \in U$, such that:

 \begin{tcolorbox}
 \begin{itemize}
    \item $V(x_0)=0$
    \item $V(x)>0, \quad x\in U-{x_0}$
    \item $\dot{V}(x)=\langle DV(x), f(x)\rangle\leq 0, \quad x\in U$ 
 \end{itemize}
 \end{tcolorbox}
 
 \begin{enumerate}
     \item  Then $x=x_0$ is \textbf{(Lyapunov) stable} and V is a \textbf{Lyapunov function}.
     \item \textbf{Asymptotic stability} if: $\dot{V}<0$
     \item \textbf{Unstable} if: 
     \begin{itemize}
         \item $\dot{V}>0$
         \item Or $V(x)$ is indefinite and $\dot{V}$ is semidefinite (psd or nsd). 
     \end{itemize}
 \end{enumerate}
