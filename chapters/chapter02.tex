\section{Fundamentals: existence, uniqueness, regularity of solutions of continuous dynamical systems}

\subsection{Peano's theorem: Existence}
This is a fundamental theorem which guarantees the existence of solutions to certain initial value problems. \danger It guarantees existence but not uniqueness! \danger

\subsection{Picard's theorem: Existence and Uniqueness}
Two assumptions need to be fulfilled, the second one being \textbf{Lipschitz continuity}, which you can intuitively think of as placing an bound on all the slopes of all the tangents you could create. 

\subsection{Geometric consequences of uniqueness}

For non-autonomous systems, intersection in phase space is possible. If you extend the phase space with variable t, then you see that you actually don't have an intersection.

\subsection{Local Vs. global existence}

\subsubsection{Theorem: continuation of solutions}
If a local solution cannot be continued up to time T, then one must have that:
$|x(t)| \rightarrow \infty$
as
$t \rightarrow \infty$

\subsubsection{Dependence on initial conditions}
\textbf{Theorem:}
If $f \in C^r_x \quad (r \geqslant 1)$ and  $ f \in C^0_t$
Then, the solution $x(t; t_0, x_0)$ is also: $C^0_{x_0}$

It turns out that the inverse of the flow map is also continuously differentiable.


