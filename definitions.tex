\section{Tricks of the nonlinear dynamic trade}

\subsection{Commonly used tricks}
\begin{itemize}
    \item Roots of quadratic polynomial: 
$$x=\frac{-b\pm\sqrt{b^2-4ac}}{2a}$$
    \item While doing a near identity transform: $\frac{1}{1+z}=1-z + )(z^2)$
    \item \textbf{Taylor series}: $$\sum_{n=0} ^ {\infty} \frac {f^{(n)}(a)}{n!} (x-a)^{n}$$
    \item Taking full derivative of a functional:
$$\frac{dL}{dt}
= \frac{\partial L}{\partial t} + \sum_{i=1}^n \frac{\partial L}{\partial x_i}\frac{dx_i}{dt}$$
    \item \textbf{Some trig identities:}
    $$\sin(\alpha \pm \beta) = \sin \alpha \cos \beta \pm \cos \alpha \sin \beta$$
    $$\cos(\alpha \pm \beta) = \cos \alpha \cos \beta \mp \sin \alpha \sin \beta$$
    \textbf{Power reducing identities:}
    $$\sin^2\theta = \frac{1 - \cos (2\theta)}{2}$$
    $$\cos^2\theta = \frac{1 + \cos (2\theta)}{2}$$
    $$\sin^2\theta \cos^2\theta = \frac{1 - \cos (4\theta)}{8}$$
    $$\sin^3\theta = \frac{3 \sin\theta - \sin (3\theta)}{4}$$
    $$\cos^3\theta = \frac{3 \cos\theta + \cos (3\theta)}{4}$$
    $$\sin^3\theta \cos^3\theta = \frac{3\sin (2\theta) - \sin (6\theta)}{32}$$
    $$\sin^4\theta = \frac{3 - 4 \cos (2\theta) + \cos (4\theta)}{8}$$
    $$\cos^4\theta = \frac{3 + 4 \cos (2\theta) + \cos (4\theta)}{8}$$
    $$\sin^4\theta \cos^4\theta = \frac{3-4\cos (4\theta) + \cos (8\theta)}{128}$$
    \textbf{Product-to-sum identities:}
    $$2\cos \theta \cos \varphi = {{\cos(\theta - \varphi) + \cos(\theta + \varphi)}}$$
    $$2\sin \theta \sin \varphi = {{\cos(\theta - \varphi) - \cos(\theta + \varphi)}}$$
    $$2\sin \theta \cos \varphi = {{\sin(\theta + \varphi) + \sin(\theta - \varphi)} }$$
    $$2\cos \theta \sin \varphi = {{\sin(\theta + \varphi) - \sin(\theta - \varphi)} }$$
   \item Linear algebra: 
\quad A =
\begin{pmatrix}
    a & b\\
    c & d
\end{pmatrix}
\begin{enumerate}
    \item \textbf{Trace}: $\operatorname{tr}(A) = \sum_{i=1}^{3} a_{ii}=a+d$
    \item \textbf{Inverse:} 
    $B^{-1} = \frac{1}{ad-cb}$
    \begin{pmatrix}
    d & -b\\
    -c & a
\end{pmatrix}
\end{enumerate}

\end{itemize}

\section{Useful definitions from lecture}
\begin{itemize}
    \item \textbf{Algebraic multiplicity of an eigenvalue} = number of times lambda appears as a root of the characteristic polynomial
    \item \textbf{Geometric multiplicity} = dimension of the eigenspace for the eigenvalue lambda. The eigenspace for lambda is given by:
    $$N(A-\lambda I)$$
    $$\text{dim(original matrix)=rank+nullity}$$
    That's how he gets that the geometric multiplicity is 1! Also note that:
    $$\text{geometric multiplicity}\leq \text{algebraic multiplicity}$$
\end{itemize}

\subsection{ODEs and PDEs}
\subsubsection{Separable differentiable equations}
You separate the $y$ and $dy$ from the $x$ and $dx$. Then you can integrate both sides. \textbf{It is usually the first thing you try!}
\begin{align*}
    &\frac{dy}{dx} = \frac{-x}{ye^{x^2}}
    \iff y\cdot dy=-xe^{x^2}dx\\
    & \int_{-\infty}^{+\infty} y\cdot dy = \int_{-\infty}^{+\infty}-xe^{x^2}dx \\
    & \iff \frac{y^2}{2} + C_1 = \frac{1}{2}e^{-x^2}+C_2
\end{align*}
This last formulation is called \textbf{the implicit form}  . And then you solve for y.

\subsubsection{Exact equations}
Consider: $$M(x,y) + N(x,y)\frac{dy}{dx} = 0$$\\
If: $$\frac{\partial M}{\partial y} = \frac{\partial N}{\partial x}$$
Then you have an \textbf{exact differential equation} and there is a $\Psi$ such that:
$$\frac{d\Psi(x,y)}{dx}=0$$
and 
$$\Psi(x,y)=C, \quad \frac{\partial \Psi}{\partial x}=M, \quad
\frac{\partial \Psi}{\partial y}=N$$

\href{https://www.khanacademy.org/math/differential-equations/first-order-differential-equations/modal/v/exact-equations-example-2}{Example from Khan Academy} to know how to solve this kind of problem.

\subsection{Second order differential equations}
\subsubsection{Linear}
$$a(x)y''+b(x)y'+c(x)y=d(x)$$

\subsubsection{Homogeneous 2nd order differential equations}
$$Ay''+By'+Cy=0$$
\begin{itemize}
    \item \textbf{Superposition applies:} If $g(x)$ and $h(x)$ are both solutions, then $a\cdotg(x)+b\cdoth(x)$ is also a solution.
    \item \textbf{Characteristic equation}: $Ar^2+Br+C=d(x)$
    \item \textbf{Most general solution:} $y(x)=C_1e^{r_1x}+C_2e^{r_2x}$
    \item \textbf{Solution for complex roots}: $r=\lambda \pm \mu i$
    $$y(x)=C_1e^{(\lambda + \mu i)x}+C_2e^{(\lambda - \mu i)x}$$
$$y(x)= e^{\lambda x}(C_1e^{\mu xi}+C_2e^{-\mu xi})$$
Given that: $e^{ix}=cosx+isinx$\\
We get:
$$y=e^{\lambda x}(C_3cos(\mu x)+C_4sin(\mu x))$$
    \item \textbf{Repeated roots:} similar to what you had before but now you need to add an x in front!
$$y(x)=C_1\cdot x \cdot e^{r_1x}+C_2e^{r_2x}$$
\end{itemize}

\subsubsection{Non-homogeneous ODE}
$$Ay''+By'+Cy=g(x)$$
You define:
\begin{itemize}
    \item \textbf{Homogeneous solution}
    \item \textbf{Particular solution}.
You can use the \textbf{method of undetermined coefficients}.
\end{itemize}
And you find a  \\
If $g(x)$ is of the form:
\begin{align}
    k e^{a x} \quad \rightarrow \quad C e^{a x}\\
    k x^n, n = 0, 1, 2 \ldots \quad \rightarrow \quad  \sum_{i=0}^n K_i x^i\\
    k \cos(a x) \text{ or } k \sin(a x) \quad \rightarrow \quad  K \cos(a x) + M \sin(a x)\\
    k e^{a x} \cos(b x) \text{ or } ke^{a x} \sin(b x) \quad \rightarrow \quad e^{a x} (K \cos(b x) + M \sin(b x))
    % \left(\sum_{i=0}^n k_i x^i\right) \cos(b x) \text{ or }\ \left(\sum_{i=0}^n k_i x^i\right) \sin(b x) \quad \rightarrow \quad \left(\sum_{i=0}^n Q_i x^i\right) \cos(b x) + \left(\sum_{i=0}^n R_i x^i\right) \sin(b x)\\
    % \left(\sum_{i=0}^n k_i x^i\right) e^{a x} \cos(b x) \text{ or } \left(\sum_{i=0}^n k_i x^i\right) e^{a x} \sin(b x) \quad \rightarrow \quad e^{a x} \left(\left(\sum_{i=0}^n Q_i x^i\right) \cos(b x) + \left(\sum_{i=0}^n R_i x^i\right) \sin(b x)\right) 
\end{align}

